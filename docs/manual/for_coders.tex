% Outline:
%
% II. For Coders
%   A. Editor Tutorial
%     1. Global Variables
%     2. MainWindow Class
%       a. handle() function
%     3. Callback Functions
%     4. main() Function
%       a. Tree items
%   B. Detailed Description
%     1. Fl_5C_Tree and _Item
%     2. Fl_5C_Popup_Window
%     3. Fl_5C_Trap_Window

\chapter{For Coders}

This chapter gives an overview of how to use the \fc{} library.  The first
section gives a tutorial on creating a simple text editor with a popup \fc{}
window.  The second section gives a more detailed description of how to use
the \texttt{Fl\_5C\_*} classes.

\section{Editor Tutorial}

This section describes how to create a simple text editor program using the
\fc{} library.  The full source code of the editor is available in the git
repository as \texttt{tests/editor.cpp}.

\subsection{Global Variables}

After all the necessary include files, several global variables are defined as
follows:

\begin{samepage}
\begin{verbatim}
class MainWindow;
MainWindow* window;
Fl_5C_Tree* tree;
Fl_Text_Editor* editor;
Fl_Text_Buffer* buffer;
string current_file;
\end{verbatim}
\end{samepage}

\subsection{MainWindow Class}

The \texttt{MainWindow} class holds an \texttt{Fl\_Text\_Editor} and is
responsible for popping up the \fc{} menu when the user right-clicks.

\begin{samepage}
\begin{verbatim}
class MyWindow : public Fl_Window
{
public:
    MyWindow(int x, int y, int w, int h, const char* l)
    : Fl_Window(x, y, w, h, l) {}

    int handle(int event);
};
\end{verbatim}
\end{samepage}

The \texttt{handle()} function is responsible for intercepting right-clicks
from the user in order to pop up the \fc{} menu.  All other events must be
forwarded to the \texttt{Fl\_Text\_Editor} so that it can provide all of our
editing functionality.

\begin{samepage}
\begin{verbatim}
int MyWindow::handle(int event)
{
    // handle right clicks and their consequences
    if (event == FL_PUSH && Fl::event_button() == 3){
        fl_popup_5c_window(tree);
        return 1;
    }
    if (Fl::pushed() == this &&
        (event == FL_DRAG || event == FL_RELEASE)){
        return 1;
    }

    // forward other events to our children
    for (int i = 0; i < children(); ++i){
        if (child(i)->handle(event)){
            return 1;
        }
    }
    return 0;
}
\end{verbatim}
\end{samepage}

\subsection{Callback Functions}

We want the editor to have cut, copy, and paste functionality so that it will
be marginally useful to users.  We also want it to be able to create new
files, open existing files, save to an already-opened file, and save to a new
file.  To get all of this functionality, we need several callback functions
for the \texttt{Fl\_5C\_Item} array that we will create later.

\subsubsection{new\_cb}

This function clears the buffer.  It is called when the user selects the "New"
menu item.

\begin{samepage}
\begin{verbatim}
void new_cb(Fl_Widget*, void*)
{
    buffer->remove(0, buffer->length());
    current_file = "";
}
\end{verbatim}
\end{samepage}

\subsubsection{open\_cb}

This function opens an existing file.  It is called when the user selects the
"Open" menu item.

\begin{samepage}
\begin{verbatim}
void open_cb(Fl_Widget*, void*)
{
    const char* chosen = fl_file_chooser("Open File", NULL, NULL);
    if (!chosen) return;
    if (buffer->loadfile(chosen) == 0) current_file = chosen;
    else fl_alert("%s", strerror(errno));
}
\end{verbatim}
\end{samepage}

\subsubsection{saveas\_cb}

This function saves to a new file.  It is called when the user selects the
"Save As" menu item.

\begin{samepage}
\begin{verbatim}
void saveas_cb(Fl_Widget*, void*)
{
    const char* chosen = fl_file_chooser("Save File", NULL, NULL);
    if (!chosen) return;
    if (buffer->savefile(chosen) == 0) current_file = chosen;
    else fl_alert("%s", strerror(errno));
}
\end{verbatim}
\end{samepage}

\subsubsection{save\_cb}

This function saves to the current file or, if there is no file open, to a new
file.  It is called when the user selects the "Save" menu item.

\begin{samepage}
\begin{verbatim}
void save_cb(Fl_Widget*, void*)
{
    if (current_file == "") saveas_cb(w, d);
    else if (buffer->savefile(current_file.c_str()) != 0){
        fl_alert("%s", strerror(errno));
    }
}
\end{verbatim}
\end{samepage}

\subsubsection{cut\_cb, copy\_cb, and paste\_cb}

These functions cut, copy, and paste selected text to and from the buffer.
They are called when the user selects the "Cut", "Copy", and "Paste" menu
items, respectively.

\begin{samepage}
\begin{verbatim}
void cut_cb(Fl_Widget*, void*)
{
    Fl_Text_Editor::kf_cut(0, editor);
}
void copy_cb(Fl_Widget*, void*)
{
    Fl_Text_Editor::kf_copy(0, editor);
}
void paste_cb(Fl_Widget*, void*)
{
    Fl_Text_Editor::kf_paste(0, editor);
}
\end{verbatim}
\end{samepage}

\subsubsection{quit\_cb}

This function exits the program.  If you want to be fancy, you can check
whether the file is currently modified and ask if the user wants to save.  It
currently does not do that.  It is called when the user selects the "Quit"
menu item.

\begin{samepage}
\begin{verbatim}
void quit_cb(Fl_Widget*, void*)
{
    exit(0);
}
\end{verbatim}
\end{samepage}

\subsection{main() Function}

The main function ties the program together.  It populates \texttt{tree} with
a \texttt{Fl\_5C\_Item} array, creates \texttt{editor}, \texttt{buffer}, and
\texttt{window}, and runs the FLTK event loop.

First, let's create the \texttt{tree}:

\begin{samepage}
\begin{verbatim}
int main(int argc, char** argv)
{
    Fl_5C_Item items[] =
    {
        {"File"},
            {"New",     0, FL_CTRL |            'n', new_cb,    0, true},
            {"Open",    0, FL_CTRL |            'o', open_cb,   0, true},
            {"Save",    0, FL_CTRL |            's', save_cb,   0, true},
            {"Save As", 0, FL_CTRL | FL_SHIFT | 's', saveas_cb, 0, true},
        {"Edit"},
            {"Cut",   0, FL_CTRL | 'x', cut_cb,   0, true},
            {"Copy",  0, FL_CTRL | 'c', copy_cb,  0, true},
            {"Paste", 0, FL_CTRL | 'v', paste_cb, 0, true},
            FL_5C_EMPTY_ITEM,
        FL_5C_EMPTY_ITEM,
        {"Quit", 0, FL_CTRL | 'q', quit_cb, 0, true},
    };
    tree = new Fl_5C_Tree(items);
    tree->loadConfig("editor.conf");
\end{verbatim}
\end{samepage}

Now let's create the editor and add it to a window:

\begin{samepage}
\begin{verbatim}
    buffer = new Fl_Text_Buffer;
    editor = new Fl_Text_Editor(0, 0, 640, 480);
    editor->buffer(buffer);
    editor->textfont(FL_COURIER);

    window = new MyWindow(0, 0, 640, 480, "5 Corners Text Editor");
    window->add(editor);
    window->end();
    window->resizable(editor);
    window->show(argc, argv);
    Fl::run();
}
\end{verbatim}
\end{samepage}

And we're done.  Edit away!

\section{Detailed Description}

This section gives a more detailed explanation of how to use the \fc{} library
classes and functions.  For a listing of everything that's anything in the
library, see the Doxygen documentation in the \texttt{docs/html/} directory.

There are 3 main classes in the library.  These are the window class
(\texttt{Fl\_5C\_Popup\_Window}) and the tree classes (\texttt{Fl\_5C\_Tree}
and \texttt{Fl\_5C\_Item}).

\subsection{Fl\_5C\_Tree and Fl\_5C\_Item}

The basic method of creating menus in \fc{} is to create an
\texttt{Fl\_5C\_Item} array.  The array is structured in a similar way to how
an \texttt{Fl\_Menu\_Item} array is structured in order to create a
traditional menu.  \texttt{Fl\_5C\_Item} is defined as follows:

\begin{samepage}
\begin{verbatim}
struct Fl_5C_Item
{
    const char* label;
    const char* shortcut_id;
    unsigned long shortcut;
    Fl_Callback* callback;
    void* user_data;
    bool leaf;
};
\end{verbatim}
\end{samepage}

This allows for the creation of static arrays of menu items in a
one-dimensional array as seen in the editor tutorial.  \texttt{label} is the
string that the user sees when the item is displayed.  \texttt{shortcut\_id}
is a simpler string that can be used in a config file.  If
\texttt{shortcut\_id} is NULL, it defaults to \texttt{label}.
\texttt{shortcut} is the key combination that activates the menu item.  It can
be a bitwise ORed combination of \texttt{FL\_CTRL}, \texttt{FL\_ALT},
\texttt{FL\_SHIFT}, and any single character.  \texttt{callback} is the
function that is called when the menu item is selected, and
\texttt{user\_data} is the parameter that is passed to \texttt{callback}.
Finally, \texttt{leaf} determines whether the item will open a submenu when
the user chooses it.

Creating an \texttt{Fl\_5C\_Tree} is simple, given such an array of
\texttt{Fl\_5C\_Item}:

\begin{samepage}
\begin{verbatim}
Fl_5C_Item items[] = {
   ...
};
Fl_5C_Tree tree(items);
\end{verbatim}
\end{samepage}

\subsection{Fl\_5C\_Popup\_Window}

\texttt{Fl\_5C\_Popup\_Window} is the main class used for opening a \fc{}
window.  However, it is usually not used directly.  Instead, the convenience
function \texttt{fl\_popup\_5c\_window} is used.  The function simply pops up
a window and calls the callback of whatever item is selected, or does nothing
if no item is selected.  The syntax is extremely simple:

\begin{samepage}
\begin{verbatim}
Fl_5C_Tree* tree;
...
fl_popup_5c_window(tree);
\end{verbatim}
\end{samepage}
