\chapter{For Users}

% Introduction

\section{Introduction and Philosophy}

In the programs that you usually use, many of the features are presented to
you in a menu bar.  It is the purpose of \fc{} to replace that system with one
that is more efficient.  Have you ever tried to sneak from a drop-down menu to
a submenu, only to activate a neighboring submenu?  It's irritating.  That's
essentially the problem we're trying to solve.

The reason there is a problem with traditional menu bars is that the clicks
which the user is required to make are difficult.  Intuition indicates that
the difficulty of making a click depends on (among other things): the distance
from the cursor's current location, and the smallness of the target.  \fc{}
attempts to exploit these observations to make each click as easy as possible.

Obviously, the way to exploit the distance observation is by setting the
distance to 0.  That is done by using right-click style actions, where a click
anywhere within the window is acceptable.

Exploiting the size-of-target idea is less obvious, but possible.  Most
operating systems have something that is accessible from a corner (e.g., the
Start menu in Windows).  Open that as fast as you can \ldots easy, isn't it?
That's because it's in a corner.  Being in a corner traps the cursor in both
of the dimensions of the screen, meaning that the size is effectively
infinite.  You can overshoot the target as much as you feel like, as long as
you're aiming at it.

So, there are 5 ways to click efficiently: in any of the four corners, or
right-clicking.  We decided right-clicking is kind of like a corner, so we
called it \fc.

By using these principles, \fc{} is able to present all of the features while
requiring only easy clicks.  It is a similar sort of improvement that many
users find by using keyboard commands as often as possible---actions are
executed in constant time, so it feels more immediate.

% Navigating the menu

\section{Navigating the Menu}

%   Starting the popup

\subsection{Opening It}

You can right-click anywhere in the program's window, and the \fc{} menu will
pop up.  It will try to place the window such that your cursor is in the
center, but it might not be possible.  For instance, if you click too far
toward an edge.

This window will trap your cursor.  Don't worry, you can get out of there by
right-clicking again.  It traps it so that you can be constrained by the
corners---that's the point of \fc.  You can now carelessly launch your mouse
in the right direction and still be in the right place.  Or you can go to it
more meekly, if you want.

%   Making the selection

\subsection{Doing Things}

When you've got a square highlighted, you can left-click anywhere in the
square.  That will do one of two things: if you've highlighted an action, it
will perform the action and close the menu.  Otherwise, you're looking at a
submenu, in which case left-clicking will drop you into that submenu.  (You
might also have tried clicking on an empty item, which looks like a black box.
That won't do anything.)

%   Getting back

\subsection{Retreating}

Anywhere within the \fc{} menu, you can right-click, and it will take you
back.  If you're in a submenu, it will take you to the parent, and if you're
in the top-level menu, it will close the window.  Much like when entering the
menu, the position of your cursor doesn't matter.

% Keyboard shortcuts

\section{Keyboard Shortcuts}

%   Usage

\subsection{Usage}

Sometimes, keyboard shortcuts are also provided in a \fc{} program.  You'll
have to check the documentation of your particular program to find out what
they are, but to execute them, you just punch in the key combination (without
having the \fc{} menu open).  It's rather obvious.

%   Configuring

\subsection{Configuring}

If your program supports it, then you can configure the keyboard shortcuts by
editing a configuration file.  They are looked for in the directory
\texttt{\symbol{126}/.5corners/}, and are called \texttt{<progname>.conf}.
They contain lines with the format \texttt{shortcut "<action>" <key
combination>}.

As an example, let's say that you've got a program called \texttt{editor}, and
you want to make ctrl-K a shortcut for copy.  What you need is a file called
\texttt{\symbol{126}/.5corners/editor.conf}, with the following contents:

\begin{verbatim}
shortcut "Copy" CTRL k
\end{verbatim}

Note that the key combination will interpret \texttt{CTRL}, \texttt{ALT} and
\texttt{SHIFT}; anything else should be a single character that indicates
which key you want.  There can only be one key besides the modifiers, and
spaces are used to separate modifiers from each other and the key.  Your
program's documentation should provide you with the labels for the actions.
